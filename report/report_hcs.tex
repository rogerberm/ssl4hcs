\RequirePackage[final]{graphicx} % to allow for graphics to show in draft mode
\documentclass[oneside, a4paper, final]{memoir} % scrreprt | memoir [twocolumn?]
\usepackage[T1]{fontenc}  % encoding, ligature removal works
\usepackage{lipsum}
\usepackage[utf8]{inputenc}
\usepackage[obeyFinal]{todonotes}
\usepackage{ifdraft}
\usepackage{amsmath}
\usepackage{microtype}
\DisableLigatures{encoding = *, family = *}
\newcommand{\TODO}[1]{\todo[size=\tiny]{#1}}
\newcommand\imglabel[1]{\textbf{\textsf{#1}}}
\renewcommand{\rmdefault}{ptm}

\setlrmarginsandblock{3cm}{2.5cm}{*}
\setulmarginsandblock{2.5cm}{2.5cm}{*}
\checkandfixthelayout

\title{
	Semi-supervised learning for phenotypic profiling of \mbox{high-content screens\ifdraft{ (DRAFT)}{\thanks{This project was held as a Lab Rotation in Computer Science,
			as required by the Master program in Computational Biology and Bioinformatics - ETH Z\"urich}}}}
\author{
	\ifdraft{Roger Bermudez-Chacon\\Supervisor: Peter Horvath}
	        {Róger Bermúdez-Chacón\thanks{Computational Biology and Bioinformatics Master program, ETH Z\"urich}\\
			 Supervisor: Peter Horvath\thanks{Light Microscopy and Screening Centre, ETH Z\"urich}}
}
\date{\today}

\begin{document}
\maketitle
<<<<<<< HEAD
\begin{abstract}
Recent advancements in high-throughput microscopy and data analysis made possible to perform large 
scale biological experiments and automatically evaluate them. For the detection of sub-cellular changes 
caused by different perturbations in the cell (RNAi or drugs) often supervised machine learning (SML) 
is used. Reliable training of an SML method requires significant effort from a field expert. The aim of 
this project is to investigate whether and to which extent intelligent control experiment design 
combined with semi-supervised learning can reach the accuracy of a human annotator and/or in certain 
cases substitute it.
\end{abstract}
=======
>>>>>>> dd52a9b2393628f364680f03af14e75fe5a5138e
\setcounter{secnumdepth}{0}
%c \part{p1}, \chapter{c1}, \section{s2}, \subsection{ss1}, \subsubsection{sss1}, \paragraph{p1}, \subparagraph{sp1}

\section{Introduction}
<<<<<<< HEAD
\lipsum[19]

\section{Method}
\lipsum[4-5]

=======
	Recent advancements in high-throughput microscopy and data analysis made possible to perform large 
	scale biological experiments and automatically evaluate them. For the detection of sub-cellular changes 
	caused by different perturbations in the cell (RNAi or drugs) often supervised machine learning (SML) 
	is used. Reliable training of an SML method requires significant effort from a field expert. The aim of 
	this project is to investigate whether and to which extent intelligent control experiment design 
	combined with semi-supervised learning can reach the accuracy of a human annotator and/or in certain 
	cases substitute it.
>>>>>>> dd52a9b2393628f364680f03af14e75fe5a5138e

\section{Data sources, Tools, and Methods}
\begin{description}
	\item[Data sources]
	labeled and unlabeled datasources
	\item[Tools] weka, numpy, scipy
	\item[methods] label spreading
\end{description}

\section{Results}
\lipsum[8]\todo{write results}

\section{Discussion}
\lipsum[9]\todo{write results}

\section{Conclusions}
\todo{fill in conclusions}
\lipsum[3]
\nocite{duda2001pattern, chapelle2006semi, hall2009weka}

\bibliographystyle{ieeetr}  % orders by occurrence in the document
\bibliography{references}
\end{document}
