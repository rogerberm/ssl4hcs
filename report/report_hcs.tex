\RequirePackage[final]{graphicx} % to allow for graphics to show in draft mode
\documentclass[oneside, a4paper, draft]{memoir} % scrreprt | memoir [twocolumn?]
\usepackage[T1]{fontenc}  % encoding, ligature removal works
\usepackage{lipsum}
\usepackage[utf8]{inputenc}
\usepackage[obeyFinal]{todonotes}
\usepackage{ifdraft}
\usepackage{amsmath}
\usepackage{microtype}
\usepackage[comma, square, super]{natbib}
\DisableLigatures{encoding = *, family = *}
\newcommand{\TODO}[1]{\todo[size=\tiny]{#1}}
\newcommand\imglabel[1]{\textbf{\textsf{#1}}}
\renewcommand{\rmdefault}{ptm}

\setlrmarginsandblock{3cm}{3cm}{*}
\setulmarginsandblock{3cm}{3cm}{*}
\checkandfixthelayout

\title{
	Semi-supervised learning for phenotypic profiling of \mbox{high-content screens\ifdraft{ (DRAFT)}{\thanks{This project was held as a Lab Rotation in Computer Science,
			as required by the Master program in Computational Biology and Bioinformatics - ETH Z\"urich}}}}
\author{
	\ifdraft{Roger Bermudez-Chacon\\Supervisor: Peter Horvath}
	{Róger Bermúdez-Chacón\\\small Computational Biology and Bioinformatics Master program\\\\
			Supervisor: Peter Horvath\\\small Light Microscopy and Screening Centre}\\\\ETH Zurich
}
\date{\today}

\begin{document}
\maketitle
\begin{abstract}
Semi-supervised machine learning techniques are particularly useful in experiments where data annotation and classification
is time- and resource-consuming or error-prone. In biological experiments this is often the case.
Here, we apply a graph-based machine learning method to classify cells in different stages of infection with the Semliki Forest Virus (SFV),
which features have been extracted from image analysis of fluorescence microscopy results, obtained in turn from a genome-wide
high-content screening experiment.
The aim of this project is to investigate whether and to which extent intelligent control experiment design 
combined with semi-supervised learning can reach the accuracy of a human annotator and/or in certain 
cases substitute it.

\end{abstract}
\setcounter{secnumdepth}{0}

\section{Introduction}
Recent advancements in high-throughput microscopy and data analysis made possible to perform large 
scale biological experiments and automatically evaluate them. For the detection of sub-cellular changes 
caused by different perturbations in the cell (RNAi or drugs), often supervised machine learning (SML) 
is used. Reliable training of an SML method, however, requires significant effort from a field expert.

As an alternative, semi-supervised machine learning (SSL) methods make use of information intrinsically found in the 
entire data, both annotated and unannotated, thus allowing to make use of a larger amount of information
by exploiting, alongside with the annotated data, the relative distribution of unannotated data on the feature space~\cite{chapelle2006semi}.
This paradigm, under a few assumptions\todo{use footnote with assumptions or just reference to publication?}\footnote{Smoothness,
Cluster, and Manifold assumptions, see~\cite[p.~4-6]{chapelle2006semi}},
has proven valuable in exploring and classifying biological data in fields as diverse as drug-protein
interactions~\cite{zheng2010semi}, gene expression~\cite{costa2007semi}, and \todo{expand this introduction?}medical diagnosis~\cite{bair2004semi}.

\section{Materials and Methods}
\subsection{High-content screening}
A human genome-wide siRNA library was used to produce human cell cultures with knocked-out
\todo{confirm this is correct/well-written, explain this in more detail?}
genes, stored in a collection of 55 16x24-well plates.
These cell cultures were exposed to a genetically engineered fluorescent SFV strand, and the corresponding
green fluorescent protein production on all of the cultures was tracked over time.

The protein expression was stopped at 4, 5, 6, and 7 hours after culture infection with SFV and microscopic pictures
of the sample were obtained under a light microscope. Samples with no exposure to SFV were also analyzed as a control
experiment, in the exact same manner as for the infected samples.
\subsection{Image adquisition and analysis}
For every sample at each infection stage, 9 tiled images were captured via a light microscope, by composing the
green fluorescent signal of the produced protein, and a blue-colored image of the nuclei. All images were
subsequently processed with an automatic random forest-based segmentation \todo{according to G. Balistreri. Confirm. Expand on this?}
tool to identify individual cells from the images.

With the mapping between microscopic pictures and individual segments representing cells, features for each cell were
extracted with CellProfiler~\cite{carpenter2006cellprofiler}.
A total of 93 features were retrieved and used in this experiment, corresponding to color intensity, area, shape, and texture descriptors.
(For a complete list of the features used, see Appendix \ref{app:featurelist})

\subsection{Annotated data}
From the genome-wide information, a small subset of the data was manually annotated by an expert on SFV infection, by visually identifying cell phenotypes
directly from the segmented microscopic images and cross-checking with the time annotation on the respective source plate,
and classifying them into the different stages of infection. This manual process yielded
3098 annotated cells.\footnote{A small number of imaging artifacts were also identified manually. However, accounting for such information was
out of the scope of this work.}

\subsection{Semi-supervised learning implementation}
A graph-based label propagation (label spreading~\cite{zhou2004learning}) approach was followed. In this kind of approach, an undirected graph is built using the
data points (cells) as vertices, and edges are created for all pairs of vertices that satisify a neighboring condition, 
with weights proportional to the degree of relatedness or association between the pair of vertices.

In the original formulation, labels are associated to the vertices corresponding to annotated data, and neutral labels to
the unannotated data; then, in an iterative fashion, the labeled vertices propagate along the edges to their neighbors' labels,
with a strength proportional to their relatedness (edge weight).

In the present implementation, prior knowledge of the nature of the data was incorporated as an additional level of \emph{soft labeling}, to exploit
the fact that, for a group of data points used as experimental control\footnote{Wild type cultures designated for plate effect monitoring, with no RNA interference applied.},
the cells (vertices) can be tracked back to their experimental conditions, which have a direct influence on what specific phenotypes (labels) are more likely to occur.

\subsubsection{Graph construction}
\textcolor{gray}{\lipsum[32]}

\subsubsection{Feature selection}
\textcolor{gray}{\lipsum[88]}

\subsubsection{Label propagation}
\textcolor{gray}{\lipsum[44]}

\section{Results}
\textcolor{gray}{\lipsum[8]}

\section{Discussion}
\textcolor{gray}{\lipsum[9]}

\section{Conclusions}
\textcolor{gray}{\lipsum[3]}

%labeled and unlabeled datasources
%	\item[Tools] weka, numpy, scipy
%	\item[methods] label spreading
%\end{description}

%\nocite{duda2001pattern, chapelle2006semi, hall2009weka, estes2007fields, banerjee2013high, carpenter2006cellprofiler}
\nocite{duda2001pattern}

\bibliographystyle{ieeetr}  % orders by occurrence in the document
\bibliography{references}
\appendix
\chapter{Features analyzed}\label{app:featurelist}
[Table with cell/nuclei intensity, shape and Haralick~\cite{haralick1973textural} texture features...]
\end{document}
